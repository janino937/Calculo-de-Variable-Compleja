\documentclass[letterpaper,twoside,11pt]{article}
\usepackage{amsmath}
\usepackage{amssymb}
\usepackage{amsthm}
\usepackage{enumerate}
\title{Takehome Exam 2}
\newtheorem{lemma}{Lemma}
\author{Jonathan Andr\'es Ni\~no Cort\'es}
\begin{document}
\maketitle
\textbf{Problem 1.} Find all possible Laurent series expansions centered at 0 of the function $\frac{1}{z^2+2z}$.

Hint: there are two ways to choose the annulus, hence two diferent Laurent series expansions.

\begin{proof}
For the first expansion we take any annulus between $r=0$ and $r=2$. Now to get the Laurent decomposition we apply partial fractions.

\begin{equation}
\frac{1}{z^2+2z}=\frac{A}{z}+\frac{B}{z+2} \implies 1=A(z+2)+Bz\nonumber
\end{equation}

When $z=0$ we have
\begin{equation}
1=2A  \implies A = \frac{1}{2} \nonumber
\end{equation}

When $z=-2$
\begin{equation}
1=-2B  \implies B = -\frac{1}{2} \nonumber
\end{equation}

So, 
\begin{equation}
\frac{1}{z^2+2z}=\frac{1}{2z}+\frac{1}{-2(z+2)} \implies 1=A(z+2)+Bz\nonumber
\end{equation}

$\frac{1}{-2(z+2)}$ is analytic on any neighborhood of radius $r<2$ arround 0, and $\frac{1}{2z}$ is analytic on any neighborhood or radius $r>0$ arround infinity. So this is indeed the Laurent descomposition of our function.

Now, to expand $\frac{1}{z^2+2z}$ we use the geometric series.

\begin{equation}
\frac{1}{-2(z+2)} = \frac{1}{-4(z/2+1)}=\frac{1}{-4(1-(-z/2))}= -\frac{1}{4}\sum_{n=0}^{\infty}(\frac{-z}{2})^n = \sum_{n=0}^{\infty}\frac{(-1)^{n+1}z^n}{2^{n+2}}\nonumber
\end{equation}

We can see that if $n=-1$ then $\frac{(-1)^{n+1}z^n}{2^{n+2}}=\frac{(-1)^{0}z^{-1}}{2^{1}}=\frac{1}{2z}$ which is our negative power term in the Laurent descomposition. Therefore,

\begin{equation}
\sum_{n=-1}^{\infty}\frac{(-1)^{n+1}z^n}{2^{n+2}} \nonumber
\end{equation}

is the first Laurent expansion.

For the second Laurent expansion we take the annulus such that it is between 2 and $\infty$. We use the change of variable $z=1/w$ to compute the expansion of the function at $\infty$.

\begin{equation}
\frac{1}{(1/w)^2+2(1/w)}= \frac{w^2}{1+2w}\nonumber
\end{equation}

Again using the geometric series

\begin{equation}
\frac{w^2}{1+2w}=\frac{w^2}{1-(-2w)}=w^2\sum_{n=0}^{\infty}(-2w)^n=\sum_{n=0}^{\infty}(-2)^nw^{n+2}= \sum_{n=2}^{\infty}\frac{(-2)^{n-2}}{z^{n}}\nonumber
\end{equation}

Finally, we can write this as

\begin{equation}
\sum_{-\infty}^{n=2}(-2)^{-n-2}z^{n}=\sum_{-\infty}^{n=2}(-2)^{n}z^{n} \nonumber
\end{equation}
\end{proof}

\newpage
\textbf{Problem 2.} Suppose $f(z)$ is analytic on the punctured plane $D = \mathbb{C}\backslash\{0\}$. Integrating its Laurent series expansion, discuss when there exists a \underline{univalued} function $g(z)$ on $\mathbb{C}\backslash\{0\}$ such
that $g'(z) = f(z)$.

\begin{proof}
Since $f(z)$ is analytic on $\mathbb{C}\backslash\{0\}$ then it is analytic on any annulus with center at 0. Therefore it has a Laurent series expansion.

This is of the form

\begin{equation}
f(z)=\sum_{-\infty}^{\infty}c_kz^k \nonumber
\end{equation}

If we integrate the terms that appear on the series we have to consider two cases.

If $k\not = -1$, then 
\begin{equation}
\int c_kz^k = \frac{c_kz^{k+1}}{k+1} + C\nonumber
\end{equation}
Which are univalued functions analytic on $\mathbb{C}\backslash\{0\}$.

If $k=-1$, then the integral $\int c_{-1}z^{-1} = c_{-1}\text{ln}(z)+ C$ which is a multivalued function.

So we conclude that the function $g$ exists when the $c_{-1}$ term in the Laurent series expansion of $f$ is equal to 0.
\end{proof}
\newpage

\textbf{Problem 3.} Find all the isolated singularities of the functions $\frac{\text{cos}(z)}{z^2-\pi^2/4}$ on $\mathbb{C} \cup \{\infty\}$
Classify them accordingly to the categories removable / pole / essential.
\begin{proof}

To analyse at the points $\pi/2$ and $-\pi/2$, we apply partial fractions to $\frac{1}{z-\pi^2/4}$.

\begin{eqnarray}
\frac{1}{z-\pi^2/4}&=&\frac{A}{z-\pi/2}+\frac{B}{z+\pi/2} \nonumber
\\ 1&=&A(z+\pi/2)+B(z-\pi/2) \nonumber
\end{eqnarray}

At $z=-\pi/2$ we have that
\begin{equation}
1=-\pi B \implies B=-1/\pi \nonumber
\end{equation} 

At $z=\pi/2$ we have that
\begin{equation}
1=\pi A \implies A=1/\pi \nonumber
\end{equation} 

So,
\begin{eqnarray}
\frac{\text{cos}(z)}{z-\pi^2/4}&=&\frac{\text{cos}(z)}{\pi(z-\pi/2)}-\frac{\text{cos}(z)}{\pi(z+\pi/2)} \nonumber
\end{eqnarray}

If we want to take the Laurent expansion of our function arround $-\pi/2$ then the left term is analytic in any neighborhood arround the point so that its expansion is a Taylor series expansion. Regarding the term of the right we have that

\begin{equation}
-\frac{\text{cos}(z)}{\pi(z+\pi/2)}= -\frac{1}{\pi(z+\pi/2)}\sum_{n=0}^{\infty}\frac{(-1)^{n+1}(z+\pi/2)^{2n+1}}{(2n+1)!}=\frac{1}{\pi}\sum_{n=0}^{\infty}\frac{(-1)^n(z+\pi/2)^{2n}}{(2n+1)!}\nonumber
\end{equation}

Note: the expansion of $\text{cos}(x)$ as shown above can be the deduce by calculating the Taylor series at $-\pi/2$. For simplicity this calculations are not shown.

As it can be seen the series has no nonnegative power terms, so the term is analytic. Finally, the sum of two analytic functions is analytic and therefore the singularity at $-\pi/2$ is a removable singularity.

Similarly, for $\pi/2$ the analytic term would be the one on the right and the left term would have series expansion of the form

\begin{equation}
\frac{\text{cos}(z)}{\pi(z-\pi/2)}= \frac{1}{\pi(z-\pi/2)}\sum_{n=0}^{\infty}\frac{(-1)^n(z-\pi/2)^{2n+1}}{(2n+1)!}=\frac{1}{\pi}\sum_{n=0}^{\infty}\frac{(-1)^n(z-\pi/2)^{2n}}{(2n+1)!}\nonumber
\end{equation}

So $\pi/2$ is also a removable singularity.

Finally, to analyse the singularity at $z=\infty$ we must take the expansion at infinity of our function. So by taking $z=1/w$ we arrive to

\begin{equation}
g(w)=\frac{\text{cos}(1/w)}{(1/w)^2-\pi^2/4} = \frac{w^2\text{cos}(1/w)}{1-w^2\pi^2/4}\nonumber
\end{equation}

We can see that the singularity at $w=0$ is isolated. If we approach 0 from the real axis we get that

\begin{equation}
\lim_{x \to 0} \frac{x^2\text{cos}(1/x)}{1-x^2\pi^2/4}= \lim_{x \to 0} x^2\text{cos}(1/x) = 0\nonumber  
\end{equation}

On the other hand, if we approach 0 from the Imaginary axis we get
\begin{eqnarray}
\lim_{+iy \to 0} -\frac{y^2\text{cos}(1/iy)}{1+y^2\pi^2/4}&=& \lim_{+iy \to 0} -y^2\text{cos}(1/iy) \nonumber
\\&=& \lim_{+iy \to 0} -y^2\frac{e^{-ii/y}+e^{ii/y}}{2} \nonumber
\\&=& \lim_{+iy \to 0} -y^2\frac{e^{1/y}+e^{-1/y}}{2} \nonumber
\\&=& \lim_{im \to \infty} -\frac{e^{m}+e^{-m}}{2m^2}\quad\text{(changing variable $1/y=m$)}\nonumber
\\&=& \lim_{im \to \infty} -\frac{e^{m}-e^{-m}}{4m} \quad\text{(L'Hospital)}\nonumber
\\&=&  \lim_{im \to \infty} -\frac{e^{m}+e^{-m}}{4} \quad\text{(L'Hospital)}\nonumber
\\&=& -\frac{\infty+0}{4}=-\infty \nonumber
\end{eqnarray}

So it can be concluded that the function behaves as an essential singularity.
\end{proof}

\newpage
\textbf{Problem 4.} Show that if $z_0$ is an isolated singularity of $f(z)$ and it is not removable, then $f(z_0)$ is an essential singularity for $e^{f(z)}$.

\begin{proof}
First assume that $z_0$ is a pole of $f$ of order $k$. Then we can write this is function as \begin{equation}f(z)=\frac{g(z)}{(z-z_0)^k} \nonumber\end{equation}.
Where $g(z)$ is analytic and $g(z_0) \not = 0$.

Now, we find two sequences $\{a_n\}$ and $\{b_n\}$ such that when they tend to zero, then $f(z_n)$ tends to $+ \infty$ and $-\infty$, respectively.

For the first one take
\begin{equation}
a_n=\frac{\sqrt[k]{-g(z_0)}}{n}+z_0 \nonumber
\end{equation} 



We can see that as $n \to \infty$, $a_n \to a_0$.

Also $f(a_n)$ would be

\begin{eqnarray}
f(a_n)&=&\frac{g(a_n)}{(\frac{\sqrt[k]{-g(z_0)}}{n}+z_0-z_0)^k} \nonumber
\\ &=&\frac{g(a_n)}{(\frac{\sqrt[k]{-g(z_0)}}{n})^k} \nonumber
\\ &=&\frac{n^k g(a_n)}{-g(z_0)} \nonumber
\end{eqnarray}

Now,
\begin{equation}
\lim_{n\to \infty}f(a_n)=\lim_{n\to \infty}\frac{n^k g(a_n)}{-g(z_0)}=\lim_{n\to \infty}\frac{n^k g(z_0)}{-g(z_0)}=\lim_{n\to \infty}-n^k = -\infty \nonumber
\end{equation}

Similarly by taking,

\begin{equation}
b_n=\frac{\sqrt[k]{g(z_0)}}{n}+z_0 \nonumber
\end{equation}

we can prove that $f(b_n) \to \infty$ as $b_n \to z_0$.

Now, to analyse the singularity at $z_0$ of $e^{f(z)}$ we consider the sequences $\{e^{f(a_n)}\}$ and $\{e^{f(b_n)}\}$. As $n \to \infty$ , $e^{f(a_n)} \to e^{-\infty}=0$, and $e^{f(b_n)} \to e^{\infty}=\infty$. So we can see that $e^{f(z)}$ behaves as an essential singularity.

Now, for the case where $z_0$ is an essential singularity then we can use Casorati-Weirstrass theorem that says we can get a sequence  $\{z_n\}$ such that as $z_n \to z_0$ then $f(z_0) \to w$ for any $w \in \mathbb{C}$.

Now, we can define a sequence that tends to $-\infty$ as follows. Let $\{f(z_mn)\}$ be a sequence such that it tends to $-n$.

By a diagonalization argument, we can take $\{f(z_ii)\}$ and this sequence will diverges to $-\infty$.

Similarly, by taking $\{f(z_mn)\}$ sequences that converge to $n$ then we can have a sequence that converges to $\infty$. Finally by the same argument as in the previous case, we have a sequence such that it converges to 0 and another such that it converges to $\infty$. Therefore the singularity behaves as an essential singularity.   
\end{proof}


\newpage

\textbf{Problem 5.} Let $f(z)$ be an analytic function on the upper half plane that is periodic,
with real period $2\pi\lambda > 0$. Suppose that there are $A,C > 0$ such tha $|f(x + iy)| \leq  Ce^{Ay}$ for $y > 0$.
Show that
\begin{equation}
\sum_{n \geq -A\lambda}a_ne^{inz/\lambda}, \nonumber
\end{equation}
where the series converges unformly in each half-plane $\{y \geq \epsilon\}$, for fixed $\epsilon > 0$.

\begin{proof}
We make the change of variables $w=e^{iz/\lambda}$, so that $f(z)=g(w)$. This function $g$ would be defined on the domain $D$, a puncture disk with center at 0 and some radius $r$. Notice that at $w=0$ the function $g$ has a singularity since this value of $w$ is never reach by a value of $z$. 

We need to prove that this function $g$ is well-defined. So take the inverse of our transformation.
\begin{eqnarray}
w&=&e^{iz/\lambda} \nonumber
\\\text{log(w)}&=&\frac{iz}{\lambda} \nonumber
\\z&=& \lambda \frac{\text{log}(w)}{i}\nonumber
\\z&=& \lambda \frac{\text{Log}(w)}{i}+\lambda \frac{i\text{arg}(w)}{i}\nonumber
\\z&=&  -i\lambda\text{Log}(w)+\lambda \text{arg}(w)\nonumber
\end{eqnarray}

Then we need to show that if we take any of the branches of the argument, $f(z)$ will still be the same. That is 
\begin{equation}
f(z)= f(-i\lambda\text{Log}(w)+\lambda \text{arg}(w)+\lambda 2 \pi)=f(-i\lambda\text{Log}(w)+\lambda \text{arg}(w)) \nonumber
\end{equation}

But since $f$ has period $2 \pi \lambda$ this condition is given.

Now by developing the expression for $w$ we can take

\begin{eqnarray}
w&=&e^{iz/\lambda} \nonumber
\\w&=&e^{i(x+iy)/\lambda} \nonumber
\\w&=&e^{(ix-y)/\lambda} \nonumber
\\w&=&e^{ix/\lambda}e^{-y/\lambda} \nonumber
\\w^{\lambda}&=&e^{ix}e^{-y} \nonumber
\\e^{-ix}w^{\lambda}&=&e^{-y} \nonumber
\end{eqnarray}

Then, the condition that $|f(x + iy)| \leq  Ce^{Ay}$ can be restated in the following way. 
\begin{eqnarray}
|f(x + iy)| &\leq&  Ce^{Ay}\nonumber
\\|g(w)| &\leq&  Ce^{Ay}\nonumber
\\e^{-Ay}|g(w)| &\leq&  C \nonumber
\\e^{-Aix}w^{A\lambda}|g(w)| &\leq&  C \nonumber
\\|e^{-Aix}w^{A\lambda}||g(w)| &\leq&  C \nonumber
\\|w^{A\lambda}g(w)| &\leq&  C \nonumber
\end{eqnarray}
Hence $h(w)=w^{A\lambda}g(w)$, its a bounded function in our domain and therefore its Laurent expansion is a normal power series expansion, meaning that $h$ can be analiticaly extended in the whole disk.

Therefore, the function
\begin{equation}
g(w)=\frac{h(w)}{w^{\lfloor A\lambda\rfloor}} \nonumber
\end{equation}

is a pole with center 0 and degree ${\lfloor A\lambda\rfloor}$. Therefore it has Laurent series expansion of the form
\begin{equation}
\sum_{k={-\lfloor A\lambda\rfloor}}^{\infty} a_kw^{k} = \sum_{k\geq {- A\lambda}}^{\infty} a_ke^{ikz/\lambda}  \nonumber
\end{equation}

Which is the result we wanted to prove.

Also we can see that $|w|=e^{-y/\lambda}$ that is $y>0$.  If we take $y>\epsilon$ then this upper half plane gets map into a puncture disk and we show that the function is analytic in that domain. Therefore it is analytic in any of the half-planes mentioned. 
\end{proof}
\newpage

\textbf{Problem 6.} Consider the continuous function $f(e^{i\theta}) = \theta^2, -\pi < \theta \leq \pi$. Find
the complex Fourier series of $f(e^{i\theta})$. Why does the series converge to the function? (except at some
points!) By substituting $\theta = 0$, show that

\begin{equation}
\frac{\pi^2}{12}=1-\frac{1}{2^2}+\frac{1}{3^2}-\frac{1}{4^2}+\cdots . \nonumber
\end{equation}

\begin{proof}
The complex Fourier series of $f(e^{i\theta})$ is of the form $\sum_{-\infty}^{\infty}c_ke^{-ik\theta}$.

To calculate each $c_k$ we must evaluate \begin{equation}\int_{-\pi}^{\pi} f(e^{i\theta})e^{-ik\theta}\frac{d\theta}{2\pi}\nonumber \end{equation}

We have two cases, if $k=0$ then 
\begin{eqnarray}
\int_{-\pi}^{\pi} f(e^{i\theta})e^{-ik\theta}\frac{d\theta}{2\pi} &=& \int_{-\pi}^{\pi} \theta^2\frac{d\theta}{2\pi}\nonumber 
\\ &=& \frac{\theta^3}{6\pi}\bigg|_{-\pi}^{\pi} \nonumber
\\ &=& \frac{\pi^3}{6\pi}-\frac{(-\pi)^3}{6\pi} \nonumber
\\ &=& \frac{\pi^2}{3}\nonumber.
\end{eqnarray}
If $k\not = 0$ then we use integration by parts
\begin{eqnarray}
\int_{-\pi}^{\pi} f(e^{i\theta})e^{-ik\theta}\frac{d\theta}{2\pi} &=& \int_{-\pi}^{\pi} \theta^2e^{-ik\theta}\frac{d\theta}{2\pi}\nonumber 
\\ &=& -\frac{\theta^2e^{-ik\theta}}{ik2\pi}\bigg|_{-\pi}^{\pi} + \int_{-\pi}^{\pi}\frac{\theta e^{-ik\theta}d\theta}{ik\pi}\nonumber
\\ &=& -\frac{\theta^2e^{-ik\theta}}{ik2\pi}\bigg|_{-\pi}^{\pi} + \frac{\theta e^{-ik\theta}}{k^{2}\pi}\bigg|_{-\pi}^{\pi}-\int_{-\pi}^{\pi}\frac{ e^{-ik\theta}d\theta}{k^2\pi} \nonumber
\\ &=& -\frac{\theta^2e^{-ik\theta}}{ik2\pi}\bigg|_{-\pi}^{\pi} + \frac{\theta e^{-ik\theta}}{k^{2}\pi}\bigg|_{-\pi}^{\pi}+\frac{ e^{-ik\theta}d\theta}{-ik^3\pi}\bigg|_{-\pi}^{\pi} \nonumber
\\ &=& 0 + (\frac{\pi e^{-ik\pi}}{k^{2}\pi}-\frac{-\pi e^{ik\pi}}{k^{2}\pi})+0 \nonumber
\\ &=& \frac{e^{-ik\pi}}{k^{2}}+\frac{e^{ik\pi}}{k^{2}} \nonumber
\\ &=& \frac{(-1)^k}{k^{2}}+\frac{(-1)^k}{k^{2}} \nonumber
\\ &=& \frac{2(-1)^k}{k^{2}}\nonumber
\end{eqnarray}

So the Fourier expansion is
\begin{equation}
\theta^2=\sum_{k=-\infty}^{-1} \frac{2(-1)^k}{k^{2}}e^{-ik\pi} +\frac{\pi^{2}}{3}+ \sum_{k=1}^{\infty} \frac{2(-1)^k}{k^{2}}e^{-ik\pi}\nonumber
\end{equation}

When we evaluate at $\theta = 0$ we get 

\begin{equation}
0=\sum_{k=-\infty}^{-1} \frac{2(-1)^k}{k^{2}}+\frac{\pi^{2}}{3}+ \sum_{k=1}^{\infty} \frac{2(-1)^k}{k^{2}}\nonumber
\end{equation}

Notice that the sum in the left is equal to the sum in the right since $\frac{2(-1)^{-k}}{(-k)^{2}}=\frac{2(-1)^k}{k^{2}}$. Therefore, we get that 

\begin{eqnarray}
\frac{\pi^{2}}{3}&=&-2\sum_{k=1}^{\infty} \frac{2(-1)^k}{k^{2}}\nonumber
\\&=&4\sum_{k=1}^{\infty} \frac{(-1)^{k+1}}{k^{2}} \nonumber
\\\frac{\pi^{2}}{12}&=&\sum_{k=1}^{\infty} \frac{(-1)^{k+1}}{k^{2}} \nonumber
\end{eqnarray}

Which is the result we want to prove.

Notice that since the function is differentiable at $(-\pi,\pi)$ then ny a theorem in Section VI.6 of the book the function converges to the series. For the point $\pi$ we cannot conclude if the series converges or not to that point. 
\end{proof}

\end{document}